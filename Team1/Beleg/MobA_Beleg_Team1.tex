\documentclass[12pt]{article}

\usepackage[ngerman]{babel}
\usepackage[utf8]{inputenc}
\usepackage{amssymb}
\usepackage{amsmath}
\usepackage{gensymb}
\usepackage{color}
\usepackage{graphicx}
\usepackage{wrapfig}
\usepackage{svg}
\usepackage{listings}
\usepackage[hidelinks]{hyperref}

\lstset{
	language=Java,
	basicstyle=\footnotesize\sffamily,
	numbers=left,
	numberstyle=\tiny,
	tabsize=2,
	columns=fixed,
	showtabs=false,
	keepspaces,
	commentstyle=\color{gray},
	keywordstyle=\color{blue}
}

\begin{document}

\begin{titlepage}
	\centering
	\includegraphics[width=0.2\textwidth]{Bilder/hszg}\par
	\vspace{1cm}
	{\LARGE Hochschule Zittau/Görlitz \par}
	\vspace{1cm}
	{\Large Mobile Anwendungen\par}
	\vspace{1.5cm}
	{\huge\bfseries FacilityManager \par}
	\vspace{0.75cm}
	{\LARGE Umsetzung als hybride Ionic-App\par}
	\vspace{2cm}
	{\Large Team 1 - Backend\par}
	\vspace{0.75cm}
	{\Large Niklas Merkelt\\Uta Lemke\par}
	\vfill
	{\large 07. Februar 2020\par}
\end{titlepage}

\tableofcontents
\newpage

\begin{abstract}
generelles Gelaber zum Projekt
\end{abstract}

\section{Aufgabenstellung}

\section{Lokale Datenbankinfrastruktur}
\subsection{Umsetzung}
\subsection{Schwierigkeiten}
\subsection{Lösungen}

\section{Kommunikation mit dem Webservice}
\subsection{Umsetzung}
\subsection{Schwierigkeiten}
\subsection{Lösungen}

\section{Objektexplorer}
\subsection{Umsetzung}
\subsection{Schwierigkeiten}
\subsection{Lösungen}

\section{Nutzerverwaltung}
\subsection{Umsetzung}
\subsection{Schwierigkeiten}
\subsection{Lösungen}

\section{Dokumentation und Code-Cleanup}
\subsection{Umsetzung}
\subsection{Schwierigkeiten}
\subsection{Lösungen}

\section{Schreibanteile}
\begin{itemize}
	\item Niklas Merkelt:
	\item Uta Lemke:
\end{itemize}

\end{document}
